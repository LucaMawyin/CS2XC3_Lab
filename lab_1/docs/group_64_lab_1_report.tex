%% Determines the type of document (including standard settings for layout).
\documentclass[letterpaper]{article}



%% Package to control the size of a page: the standard margins used by LaTeX are
%% very wide!
\usepackage[margin=1in]{geometry}


%% Packages add functionality to the document. The AMS packages are standard
%% packages to support various mathematical notations. AMS stands for American
%% Mathematical Society, the organization that maintains these packages.
\usepackage{amsmath,amsthm,amssymb}

%% Theorems-like environments using functionality provided by amsthm.
\theoremstyle{plain}
\newtheorem{theorem}{Theorem}[section]
    %% [section] at the end specifies that theorems should be numbered
    %% per-section: Section x starts with theorem-like x.1 and so on, ...
\newtheorem{proposition}[theorem]{Proposition}
    %% [theorem] in the middle specifies: use the same counter as the theorem
    %% environment: here we number all theorem-like environments consecutively.
\newtheorem{corollary}[theorem]{Corollary}
\newtheorem{lemma}[theorem]{Lemma}
\theoremstyle{definition}
\newtheorem{definition}[theorem]{Definition}
\theoremstyle{remark}
\newtheorem{example}[theorem]{Example}
\newtheorem{remark}[theorem]{Remark}


%% Support for nicely formatted tables.
\usepackage{booktabs}


%% Support for colors & colors in tables.
\usepackage[table]{xcolor}

% Seven colors safe for use color blindness.
% Colors taken from doi:10.1038/nmeth.1618.
\definecolor{cbOrange}{RGB}{230,159,0}
\definecolor{cbSkyBlue}{RGB}{86,180,233}
\definecolor{cbBluishGreen}{RGB}{0,158,115}
\definecolor{cbYellow}{RGB}{240,228,66}
\definecolor{cbBlue}{RGB}{0,114,178}
\definecolor{cbVermillion}{RGB}{213,94,0}
\definecolor{cbReddischPurple}{RGB}{204,121,167}


%% Notation used in this document.
\newcommand{\n}{\mathbf{n}} %% Num. Replicas.
\newcommand{\f}{\mathbf{f}} %% Num. Faulty Replicas.

%% Misc. Math notation.
\newcommand{\BigO}{\mathcal{O}}
\newcommand{\abs}[1]{\lvert #1 \rvert}
\newcommand{\AName}[1]{\textsc{#1}}
\newcommand{\Var}[1]{\texttt{#1}}


%% Algorithms.
\usepackage{algorithm}
\usepackage[noend]{algorithmic}
\newcommand{\GETS}{:=}


%% Formatting SI-units.
\usepackage{siunitx}
\sisetup{per-mode=symbol}


%% TikZ: for creating figures.
\usepackage{tikz}

%% Configuration for figures: Nicer arrows.
\usetikzlibrary{arrows.meta}
\tikzset{>=Stealth}


%% pgfplots: drawing plots using TikZ.
\usepackage{pgfplots}
%% Configuration for plots: Use color-blind friendly colors.
\pgfplotscreateplotcyclelist{cbSafeList}{
    very thick,solid,cbOrange,every mark/.append style={solid},mark=*\\
    very thick,solid,cbSkyBlue,every mark/.append style={solid},mark=*\\
    very thick,solid,cbBluishGreen,every mark/.append style={solid},mark=*\\
    very thick,solid,cbYellow,every mark/.append style={solid},mark=*\\
    very thick,solid,cbBlue,every mark/.append style={solid},mark=*\\
    very thick,solid,cbVermillion,every mark/.append style={solid},mark=*\\
    very thick,solid,cbReddischPurple,every mark/.append style={solid},mark=*\\
    very thick,solid,black,every mark/.append style={solid},mark=*\\
}
\pgfplotsset{
    legend style={font=\small},
    compat=1.16,
    width=260pt,
    height=140pt,
    legend cell align=left,
    xlabel near ticks,
    ylabel near ticks,
    every axis/.append style={
        cycle list name=cbSafeList,
        ymin=0,
        enlargelimits=0.05,
        mark size=1pt,
        ylabel style={align=center},
        xlabel style={align=center},
        title style={align=center}
    }
}

%% PgfplotsTable: loading data files to use with pgfplots.
\usepackage{pgfplotstable}


%% Support for hyperlinks and urls. The setting ``colorlinks'' sets how links
%% are shown in the document (with a color, without underline). We put hyperref
%% last---it has a tendency to break other packages when loaded before them.
\usepackage[colorlinks]{hyperref}
\usepackage{graphicx}
\usepackage{pgffor}
\usepackage{caption}
\usepackage{tabularx}
\usepackage{enumitem}
\usepackage{fancyhdr}
\usepackage{float}

\title{Assignment}
\author{Luca Mawyin - 400531739}
\date{\today}

\begin{document}

\begin{titlepage}

    \thispagestyle{empty}
    \centering
    \vspace*{3em}

    \Huge{COMPSCI 2XC3 Lab Report 1}\\[1em]
    \Large Version 1.0\\[0.5em]
    \Large Prepared by\\[2em]
    \Large Group 64\\[2em] 

    \LARGE Luca Mawyin\\[0.5em]
    \LARGE Anderson Ray\\[0.5em]
    \LARGE Theo Pham\\[2em]
    
    \Large COMPSCI 2ME3 \\[0.5em]
    \Large McMaster University\\[2em]     
    \Large \today\\[2cm]

\end{titlepage}

\stepcounter{section}
\section*{Experiment \thesection}
In our experiment comparing Bubble Sort, Insertion Sort, and Selection Sort. We ran 100 tests for each sorting algorithm going from a list length of 0, to a list length of 1000 each list length being 10 elements longer then the last.

\begin{figure}[h]
    \centering
    \includegraphics[width=0.8\textwidth]{Figure_1.png}
    \caption{List Length vs. Time (ms) for Bubble, Selection, and Insertion Sort.}
    \label{fig:sort_graph}
\end{figure}

Looking at the slopes, each algorithm looks to have a parabolic shape which makes sense as we know that the algorithms used are $O(n^2)$.

Bubble sort is the slowest as the inner loop, loops through the entire list every iteration of the outer loop. Selection sort is faster then Insertion sort as swaps elements in the list smarter using less memory reads and writes making it faster.
\stepcounter{section}
\section*{Experiment \thesection}

\stepcounter{section}
\section*{Experiment \thesection}
In our experiment comparing Bubble Sort, Insertion Sort, and Selection Sort on sorted lists with varying numbers of swaps made. We ran 109 tests for each sorting algorithm on lists of size 2000 with the number of swaps ranging from 0 - 10965. Each test would increase the number of swaps made by 100.

\begin{figure}[H]
    \centering
    \includegraphics[width=0.8\textwidth]{Figure_3.png}
    \caption{List Length vs. Number of Swaps on a Sorted List for Bubble, Selection, and Insertion Sort.}
    \label{fig:sort_graph_swaps}
\end{figure}

Looking at the graph, Bubble Sort and Insertion Sort preform better with less swaps. Bubble Sort preforms better with less swaps, as its doing much less memory reads and writes in the innerloop. Insertion swap preforms better with less swaps as it exits out the second loop if it runs into sorted pairs of elements which are more common with less swaps.

Selection Sort doesn't perform better with less swaps as the number of checks and swaps are independent of whether the list is sorted or not. Even if the innerloop can't find a min index past L[i] it still swaps L[i] with itself.
\stepcounter{section}
\section*{Experiment \thesection}

\stepcounter{section}
\section*{Experiment \thesection}

\stepcounter{section}
\section*{Experiment \thesection}

\stepcounter{section}
\section*{Experiment \thesection}

\stepcounter{section}
\section*{Experiment \thesection}

\end{document}