%% Determines the type of document (including standard settings for layout).
\documentclass[letterpaper]{article}



%% Package to control the size of a page: the standard margins used by LaTeX are
%% very wide!
\usepackage[margin=1in]{geometry}


%% Packages add functionality to the document. The AMS packages are standard
%% packages to support various mathematical notations. AMS stands for American
%% Mathematical Society, the organization that maintains these packages.
\usepackage{amsmath,amsthm,amssymb}

%% Theorems-like environments using functionality provided by amsthm.
\theoremstyle{plain}
\newtheorem{theorem}{Theorem}[section]
    %% [section] at the end specifies that theorems should be numbered
    %% per-section: Section x starts with theorem-like x.1 and so on, ...
\newtheorem{proposition}[theorem]{Proposition}
    %% [theorem] in the middle specifies: use the same counter as the theorem
    %% environment: here we number all theorem-like environments consecutively.
\newtheorem{corollary}[theorem]{Corollary}
\newtheorem{lemma}[theorem]{Lemma}
\theoremstyle{definition}
\newtheorem{definition}[theorem]{Definition}
\theoremstyle{remark}
\newtheorem{example}[theorem]{Example}
\newtheorem{remark}[theorem]{Remark}


%% Support for nicely formatted tables.
\usepackage{booktabs}


%% Support for colors & colors in tables.
\usepackage[table]{xcolor}

% Seven colors safe for use color blindness.
% Colors taken from doi:10.1038/nmeth.1618.
\definecolor{cbOrange}{RGB}{230,159,0}
\definecolor{cbSkyBlue}{RGB}{86,180,233}
\definecolor{cbBluishGreen}{RGB}{0,158,115}
\definecolor{cbYellow}{RGB}{240,228,66}
\definecolor{cbBlue}{RGB}{0,114,178}
\definecolor{cbVermillion}{RGB}{213,94,0}
\definecolor{cbReddischPurple}{RGB}{204,121,167}


%% Notation used in this document.
\newcommand{\n}{\mathbf{n}} %% Num. Replicas.
\newcommand{\f}{\mathbf{f}} %% Num. Faulty Replicas.

%% Misc. Math notation.
\newcommand{\BigO}{\mathcal{O}}
\newcommand{\abs}[1]{\lvert #1 \rvert}
\newcommand{\AName}[1]{\textsc{#1}}
\newcommand{\Var}[1]{\texttt{#1}}


%% Algorithms.
\usepackage{algorithm}
\usepackage[noend]{algorithmic}
\newcommand{\GETS}{:=}


%% Formatting SI-units.
\usepackage{siunitx}
\sisetup{per-mode=symbol}


%% TikZ: for creating figures.
\usepackage{tikz}

%% Configuration for figures: Nicer arrows.
\usetikzlibrary{arrows.meta}
\tikzset{>=Stealth}


%% pgfplots: drawing plots using TikZ.
\usepackage{pgfplots}
%% Configuration for plots: Use color-blind friendly colors.
\pgfplotscreateplotcyclelist{cbSafeList}{
    very thick,solid,cbOrange,every mark/.append style={solid},mark=*\\
    very thick,solid,cbSkyBlue,every mark/.append style={solid},mark=*\\
    very thick,solid,cbBluishGreen,every mark/.append style={solid},mark=*\\
    very thick,solid,cbYellow,every mark/.append style={solid},mark=*\\
    very thick,solid,cbBlue,every mark/.append style={solid},mark=*\\
    very thick,solid,cbVermillion,every mark/.append style={solid},mark=*\\
    very thick,solid,cbReddischPurple,every mark/.append style={solid},mark=*\\
    very thick,solid,black,every mark/.append style={solid},mark=*\\
}
\pgfplotsset{
    legend style={font=\small},
    compat=1.16,
    width=260pt,
    height=140pt,
    legend cell align=left,
    xlabel near ticks,
    ylabel near ticks,
    every axis/.append style={
        cycle list name=cbSafeList,
        ymin=0,
        enlargelimits=0.05,
        mark size=1pt,
        ylabel style={align=center},
        xlabel style={align=center},
        title style={align=center}
    }
}

%% PgfplotsTable: loading data files to use with pgfplots.
\usepackage{pgfplotstable}


%% Support for hyperlinks and urls. The setting ``colorlinks'' sets how links
%% are shown in the document (with a color, without underline). We put hyperref
%% last---it has a tendency to break other packages when loaded before them.
\usepackage[colorlinks]{hyperref}
\hypersetup{
    linkcolor=blue,
    citecolor=blue,
    urlcolor=blue
}
\usepackage{tocloft}
% Make Table of Contents and List of Figures text blue
\renewcommand{\cftsecfont}{\color{blue}}
\renewcommand{\cftsecpagefont}{\color{blue}}

\renewcommand{\cftfigfont}{\color{blue}}
\renewcommand{\cftfigpagefont}{\color{blue}}
\usepackage{graphicx}
\usepackage{pgffor}
\usepackage{caption}
\usepackage{tabularx}
\usepackage{enumitem}
\usepackage{fancyhdr}
\usepackage{float}

%% Page numbering
\pagestyle{fancy}
\fancyhead[R]{\textit{Page \thepage}}
\fancyhead[L]{\textit{LAB REPORT 2}}
\fancyfoot[C]{}
\renewcommand{\headrulewidth}{0pt}
\renewcommand{\footrulewidth}{0pt}

\renewcommand{\thesubsection}{\arabic{subsection}}

\begin{document}

\begin{titlepage}

    \thispagestyle{empty}
    \centering
    \vspace*{3em}

    \Huge{COMPSCI 2XC3 Lab Report 2}\\[1em]
    \Large Prepared by\\[2em]
    \Large Group 64\\[2em] 

    \LARGE Luca Mawyin\\[0.5em]
    \LARGE Anderson Ray\\[0.5em]
    \LARGE Theo Pham\\[2em]
    
    \Large COMPSCI 2ME3 \\[0.5em]
    \Large McMaster University\\[2em]     
    \Large \today\\[2cm]

\end{titlepage}

\tableofcontents
\newpage

\listoffigures
\newpage


\stepcounter{section}
\section*{Part \thesection}
\addcontentsline{toc}{section}{Part \thesection}

\stepcounter{subsection}
\subsection*{Experiment \thesubsection}
\label{exp1}
\addcontentsline{toc}{subsection}{Experiment \thesubsection: TODO}

In our experiment finding probabilties of cycles existing in graph, we used the following parameters:
\begin{itemize}
\item  200 Nodes
\item  ranging from 0 to 100 edges
\item  running 10000 tests per edge amount
\end{itemize}

\begin{figure}[H]
    \centering
    \includegraphics[width=0.8\textwidth]{figures/Figure_\thesubsection.png}
    \caption{Probabilties of cycles existing in graphs with 200 nodes}
    \label{fig:graph_\thesubsection}
\end{figure}

In \hyperref[fig:graph_\thesubsection]{Figure~\ref{fig:graph_\thesubsection}} at around 40 edges, the graph tops out meaning that it's very likely that a graph with 200 nodes and over 40 random edges will have a cycle. This is because each time a random edge is added, the chances of the edge creating a cycle increases, resulting in the S shaped curve in \hyperref[fig:graph_\thesubsection]{Figure~\ref{fig:graph_\thesubsection}}.

\stepcounter{subsection}
\subsection*{Experiment \thesubsection}
\label{exp2}
\addcontentsline{toc}{subsection}{Experiment \thesubsection: TODO}
In our experiment finding probabilties of graphs being connected, we used the following parameters:
\begin{itemize}
\item 200 Nodes
\item ranging from 0 to 1500
\item doing 1000 tests every 10 edge amounts (0 edges, 10 edges, 20 edges, ..., 1000 edges )
\end{itemize}
\begin{figure}[H]
    \centering
    \includegraphics[width=0.8\textwidth]{figures/Figure_\thesubsection.png}
    \caption{Probabilties of cycles existing in graphs with 100 nodes}
    \label{fig:graph_\thesubsection}
\end{figure}

In \hyperref[fig:graph_\thesubsection]{Figure~\ref{fig:graph_\thesubsection}} at around 1000 edges, the graph tops out meaning that it's very likely that a graph with 200 nodes and over 1000 random edges will be connected. The graph with the max amount of edges possible without making it connected would look like a graph where every node is connected to every other node except for one specific node. To get this max number of edges, you set $N = \text{number of nodes}$ and plug it into $\frac{(N-1)(N-2)}{2}$, which gives us $19701$ for our experiment. Our experiment tops out much before this max though as the chances of fully connecting the entire graph with a single unique edge increase as the graph has more and more existing edges. This makes it very unlikley for more denser graphs($\geq 1000\text{ edges}$) to be unconnected.

In \hyperref[fig:graph_\thesubsection]{Figure~\ref{fig:graph_\thesubsection}} the graph bottoms out until about 300 edges. This is because a graph cannot be connected if: $\text{number of edges} < \text{number of nodes} - 1$ and even if you have more edges then that if they are picked at random it's very unlikely for it to be connected. 


\newpage
\stepcounter{section}
\section*{Part \thesection}
\addcontentsline{toc}{section}{Part \thesection}

\end{document}